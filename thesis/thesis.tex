% !TEX encoding = UTF-8 Unicode
% !TEX TS-program = pdflatex

% toptesti document class
\documentclass[%
    a4paper, % not needed, by default it is a4paper, or also b5paper can be used
    corpo=12pt, % dimension of basic font
    % oneside is generally the way to go
    oneside, % two side optimizes for two-face printing, having chapters open on the right (aka odd numbers), if you don't want blank pages put oneside here
    stile=standard,
    %evenboxes, % not needed, to put supervisors and candidate at the same level
    tipotesi=magistrale,
    numerazioneromana, % roman numbering for appendixes and preambles, up to Table of Contents
    openright, % to force opening on the right for double-sided printing
    cucitura=7mm, % for printing, 7mm should be enough
    %dvipsnames, % for compatibility with xcolor, it does not work
]{toptesi}

\input{common/packages}

\input{common/package_config}

\input{common/new_commands}

\input{common/thesis_info.tex}

\input{common/font_config.tex}

\addbibresource{bibliography.bib}

% to load the glossaries, not needed if using bib2gls
\input{glossaries.tex}
\makeglossaries

\begin{document}

\input{common/config}

\input{common/toptesi_config}

% front page
% frontespizio can be used for the first page print
% while the custom-made frontpage can be used as hard-cover
% use pdfjoin or pdfseparate to extract or put together the pages if needed
%\frontespizio* % without star the logo is on top
\input{common/frontpage.tex} % custom frontpage
%\retrofrontespizio
% insert text for the back of the front page
% if you insert any remove the following \paginavuota
% either a blank page or a back is needed to have double-sided printing
% pay attention to leave the space for the page

%\paginavuota % clears a page

\frontmatter

% abstract if needed
% \begin{abstract}
%     \input{content/abstract.tex}
% \end{abstract}

% to create blank pages for openright in frontmatter
% use one of the following two methods
% 1) use the following three lines
%\phantom{0} % needed otherwise cleardoublepage does not clean the page because it sees it empty
%\cleardoublepage
%\thispagestyle{empty} % to have empty page, without numbers
% 2) or
\paginavuota % to manually create a blank page

\sommario
\input{content/summary}

\phantom{0}
\cleardoublepage
\thispagestyle{empty}

\ringraziamenti % acknowledgements
\input{content/acknowledgements}

\paginavuota
\tableofcontents

\listoftables % ToC for tables

\listoffigures % ToC for figures

% actually abbreviation is the name used for acronym in glossaries-extra
% title sets the name
% type tells the type of glossary to print
% style overrides the global style
% here we are printing only abbreviations
% printunsrtglossary if using record, otherwise printglossary is ok
\paginavuota
\printunsrtglossary[style=altlist,title=Acronyms,type=\glsxtrabbrvtype]

% also list of symbols here if needed

\input{common/post_summary_config.tex}

\mainmatter

%\part{Prima Parte} % parts division, not needed
% Chapters always open on a right-side page, i.e. odd numbers, so a blank page is inserted if needed
%\cleardoublepage[empty] % to have a fully blank page
% a blank page appears before the first chapter in some configurations, on the last version it doesn't

\input{content/chapters.tex}

% \paginavuota % it works even without stile=classica

\appendix
\input{content/appendix}

% endnotes here if needed

\phantom{0}
\cleardoublepage
\printbibliography[heading=bibintoc] % heading required to show it in ToC

\end{document}